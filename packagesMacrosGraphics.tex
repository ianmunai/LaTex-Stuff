\documentclass[10pt,letterpaper]{article}
\usepackage{amsmath, amssymb, float, amsfonts,enumerate,graphicx}
% install this fullpage package - gives a 1 inch margin to the whole document
% \usepackage{fullpage}
% does not add 1 inch margin all through. gives more control
\usepackage[margin=1in]{geometry}

\parindent 0px

\begin{document}
\textbf{Critical Thinking Questions}

% This is how to insert a figure in your document
\begin{figure}[H]
    \centering
    \includegraphics[width=0.15\textwidth]{poster!.jpg}\\
    \caption{The Squeeze Theorem}
\end{figure}

\begin{enumerate}
    \item This is the symbol for the set of all real numbers: $\mathbb{R}$.
    \item This is the symbol for the set of integers: $\mathbb{Z}$.
    \item This is the symbol for the set of rational numbers: $\mathbb{Q}$.
    \item Is it possible for a sequence to converge to two differnet numbers? If so, give an example, if not, explain why not.
    \item Explain why $\int\limits_{1}^{\infty}f(x)\,dx$ and $\sum\limits_{n=1}^{\infty} a_n$ need not converge to the same value, even if they are both convergent.
    \item In your own words, explain the Alternating Series Remainder Theorem. How is this theorem useful?
    \item Explain the difference between absolute and conditional convergence. Give and example of each.
    \item The ratio test is inconclusive if $\displaystyle{\lim\limits_{n \to \infty}\left|\frac{a_{n+1}}{a_n}\right| = 1}$. Give an example of one convergent series and one divergent series for which $\displaystyle{\lim\limits_{n \to \infty}\left|\frac{a_{n+1}}{a_n}\right| = 1}$. Explain how you determined your examples.
\end{enumerate}

\begin{enumerate}
    \item $\lim\limits_{n \to \infty}$
    \item $\sum\limits_{n=1}^{\infty}$
    \item $\int\limits_{6}^{10}$
\end{enumerate}

%do not use math mode in align. It is automatically in math mode.
\begin{align}
    x^2+2x+c
\end{align}

\end{document}